\documentclass[a4paper,12pt]{article}
\usepackage{graphicx}
\usepackage[dvipsnames]{xcolor}
\usepackage{wrapfig}
\usepackage{color}
\usepackage{graphicx}
\usepackage{subfigure}
\usepackage{multirow}
\usepackage[section] {placeins}
\usepackage{tikz}
\usepackage{amsmath}
\usetikzlibrary{matrix,calc}
\usetikzlibrary{positioning}
\usetikzlibrary{matrix}
\pgfdeclarelayer{background}
\pgfsetlayers{background,main}
\usepackage[hidelinks]{hyperref}
\usepackage[nottoc]{tocbibind}

\definecolor{darkred}{rgb}{0,0,0.5}
\definecolor{darkgreen}{rgb}{0,0.5,0}
\definecolor{darkblue}{rgb}{0.5,0,0}
\hypersetup{ colorlinks, linkcolor=darkblue, filecolor=darkgreen, urlcolor=darkred, citecolor=darkblue}

\begin{document}
\begin{titlepage}
\begin{center}
	\textsc{\LARGE Indian Institute of Technology
			\\Bombay} \\[2.5cm]
	\begin{figure}[ht!]
	\begin{center}
		\includegraphics[scale = 1.4]{images/1}
	\end{center}
	\end{figure}

    \vspace*{-0.6cm}
	\hrule \hrule \hrule
	\vspace{0.5cm}	
	\textsc{\Large CS 492 : BTP Stage I}\\[0.5cm]
	% Title
	{\huge \bfseries Erlang Distributed File System (eDFS)} \\[0.5cm]
	\hrule \hrule \hrule
	\vspace{4cm}
	
	% Author and supervisor
	\begin{minipage}{0.4\textwidth}
	\begin{flushleft} \large
		\emph{By:} \\
		Aman Mangal (100050015)
	\end{flushleft}
	\end{minipage}
	\begin{minipage}{0.4\textwidth}
	\begin{flushright} \large
		\emph{Coordinator:} \\
		Prof. G. Sivakumar
	\end{flushright}
	\end{minipage}
	\vspace{0.5cm}
	
	% Bottom of the page
	{\large \today}
\end{center}
\end{titlepage}
\tableofcontents
\vspace{0.5cm}
\hrule \hrule \hrule
\newpage

\section*{Abstract}
It is extremely difficult for single system to provide high availability, fast response, large data storage and low cost all at the same time. We need multiple systems running in parallel working closely together. We call such systems \textit{Distributed Systems}. File systems are key component to store data and we, therefore, need Distributed File System (DFS). It satisfies the needs of applications that process large volumes of data such as search engines, data mining applications etc.

Erlang is recently developed general purpose concurrent functional programming language. It was designed by Ericsson to support distributed, fault tolerant, scalable, non-stop applications. It has been used in production systems (e.g. AXD301 ATM switch) with an uptime percentage of 99.9999999\% (nine nine's) \cite[p.~170]{armstrong} \cite{blog_joe}. It is being used by \textbf{Facebook, Github, Riak, Amazon} etc. to develop large distributed systems.

We have leveraged the distributed capabilities of Erlang and have developed yet another DFS namely Erlang Distributed File System (eDFS). We describe the architecture of eDFS in this report. We also look at history of DFSs and compare with eDFS. The main focus has been scalability, efficiency and simplicity while developing the file system. Map reduce has also been implemented for the purpose of testing. A lot of work is still to be done.

\section{Background}
Distributed File System is an extension of File System which manages files and data on multiple storage devices and provides more performance and reliability using various modern techniques. Outside world only sees it as a single storage device and thus simplifying the interface to a great extent. It also provides location transparency and redundancy to improve data availability in case of failure or heavy load.

The idea of eDFS is to use Erlang distributive capabilities and build a fault tolerant, highly scalable and reliable distributed file system. We keep in mind the following assumptions while designing eDFS just like Google File System (GFS) \cite{ghemawat03} -
\begin{enumerate}
\item Component failures are norm rather than exception.
\item Files are huge by traditional standards.
\item Most files are mutated by appending rather than overwriting existing data, we may support concurrent append in future
\item Sequential access are more compared to random access. We optimize the file system for stream access.
\end{enumerate}

We prefer performance over security as of now. We support the usual operations to create, delete, open, close, read and write files.

\section{History of DFS}

\section{Motivation}

\section{Design Overview}
\subsection{Architecture}
A cluster of eDFS contains a master node, many worker nodes and one or more than one client servers as shown in figure \ref{edfs_design} on page \pageref{edfs_design}. Client communicate with client server in order to perform operations on file system. Client can be a browser or any other application which can communicate over tcp/ip connection. Client server can directly communicate to master node or any worker node. Multiple clients can perform operations on file system at the same time. Multiple client servers can be deployed for load balancing.
\begin{figure}[h]
  \begin{center}
    \includegraphics[scale = 0.5]{images/design}
  \end{center}
  \caption{eDFS design}
  \label{edfs_design}
\end{figure}

\subsection{Master node (Metadata Server)}
Master node takes care of handling metadata. It is stored in mnesia in ets tables only. Each file is divided into chunks of equal size. Every chunk is assigned a unique id and stored on multiple worker node based on the replication factor of the file. Following is the table sotred in mnesia-
\begin{table}[h]
\centering
\begin{tabular}{|c|}
\hline 
name \\ 
\hline 
replication factor \\ 
\hline 
chunks \\ 
\hline 
\end{tabular} 
\end{table}

\begin{itemize}
\item \textbf{Name:} name of the file
\item \textbf{Replication Factor:} number of replicas of each chunk
\item \textbf{Chunks:} list of all the chunks of the file stored as \{id, byte begin, byte end, replicas\}
\item \textbf{Id:} chunk id
\item \textbf{Byte begin:} byte number of first byte in the chunk
\item \textbf{Byte end:} byte number of last byte in the chunk
\item \textbf{Replicas:} list of worker ids where replication of the chunk are stored
\end{itemize}

We may support multiple master node and replication of metadata in future.

\subsection{Generation of Chunk Id}
Chunk id is a unique randomly generated string. The chunk is stored with the same name on every worker node. The name can only contain letters a-z, A-Z, 0-9, ".", "\_" (64 letters). It is assumed that name is calculated with less than a million per sec frequency.  Timestamp from operating system is converted into an equivalent representation of a random string and used as the name of a chunk. It is represented using 8 letters (~64 bits). It is possible to generate such ids upto year of 2170 which is approximately 200 years later than the time since when cpu counts the number of seconds (1970).

\subsection{Worker Node}

\section{System Interaction}
\subsection{OTP Hierarchy}
\subsection{Data Flow}
\subsection{Data Replication}
\subsection{Garbage Collection}

\section{Future Work}

\section{Testing}
\subsection{Map Reduce}

\begin{thebibliography}{99}
\bibitem{ghemawat03}
  \textsc{Sanjay Ghemawat, Howard Gobioff and Shun-Tak Leung},
  "The Google file system",
  \emph{In SOSP '03: Proceedings of the nineteenth ACM symposium on Operating systems principles}
  New York, NY, USA, 2003.

\bibitem{hadoop}
  \textsc{Konstantin Shvachko, Hairong Kunag, Sanjay Radia and Robert Chansler},
  "The Hadoop Distributed File System"
  Sunnyvale, California USA

\bibitem{armstrong}
  \textsc{Joe Armstrong},
  "Making Reliable Distributed Systems in the Presence of Software Errors",
  \emph{A Dissertation submitted to the Royal Institute of Technology Stockholm}
  Sweden, December 2003.

\bibitem{douceur}
  \textsc{John Douceur, Roger Wattenhofer},
  "Optimizing file availability in a server-less distributed file system"
  \emph{In Proceedings of the 20th Symposium on Reliable Distributed Systems},
  2001.

\bibitem{eliezer}
  \textsc{Eliezer levy, Abraham silberschatz},
  "Distributed File Systems: Concepts and Examples",
  \emph{ACM Computing Surveys, Vol. 22, No. 4},
  December 1990.

\bibitem{saito}
  \textsc{Yasushi Saito and Marc Shapiro},
  "Optimistic Replication",
  \emph{ACM Computing Surveys, Vol. 37, No. 1},
  March 2005, pp. 42-81.

\bibitem{satya}
  \textsc{Satyanarayanan, M.},
  "A Survey of Distributed File Systems",
  \emph{Technical Report CMU-CS-89- 116, Department of Computer Science, Camegie Mellon University},
  1989.

\bibitem{nelson}
  \textsc{Nelson, M.N., et al},
  "Caching in the Sprite Network File System",
  \emph{ACM Transactions on Computer Systems},
  February, 1988.

\bibitem{rowe}
  \textsc{Rowe, L.A., Birman, K.P.},
  "A Local Network Based on the Unix Operating System",
  \emph{IEEE Transactions on Software Engineering SE-8(2)},
  March, 1982.

\bibitem{edmund}
  \textsc{Edmund B. Nightingale, Peter M. Chen, and Jason
Flinn},
  "Speculative Execution in a Distributed File System",
  \emph{ACM SOSP’05},
  October 23–26, 2005, Brighton, United Kingdom.

\bibitem{howard}
  \textsc{Howard J. H., Kazar M. L., Menees S. G., Nichols D. A.,
Satyanarayanan M., Sidebotham R. N. and West M. J.}
  "Scale and performance in a distributed file system",
  \emph{ACM Transactions on Computer Systems, Vol. 6, Issue1},
  February 1988

\bibitem{panasas}
  \textsc{Nagle D., Serenyi D., Matthews A.},
  "The Panasas ActiveScale Storage Cluster: Delivering Scalable High Bandwidth Storage",
  \emph{Proceedings of the 2004 ACM/IEEE conference on Supercomputing, pp. 53-},
  2004.

\bibitem{yu}
  \textsc{Yu W., Liang Sh., Panda D.K.},
  "High Performance Support of Parallel Virtual File System (PVFS2) over Quadrics",
  \emph{Proceedings of the 19th annual international conference on Supercomputing, pp. 323-331},
  2005.

\bibitem{chandramohan}
  \textsc{Chandramohan A. Thekkath, et al},
  "Frangipani: A Scalable Distributed File System",
  \emph{System Research Center, Digital Equipment Corporation},
  Palo Alto, CA, 1997.

\bibitem{liskov}
  \textsc{Barbara Liskov, et al},
  "Replication in the Harp File System",
  \emph{Laboratory of Computer Science, MIT, Cambridge},
  CA, 1991.

\bibitem{blog_joe}
  \textsc{Joe Armstrong},
  "What's All the Fuss About Erlang",
  \emph{http://pragprog.com/articles/erlang},
  2007.
\end{thebibliography}

\end{document}
